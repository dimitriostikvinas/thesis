\chapter{Επίλογος} \label{Chapter4}
Ανακεφαλαιώνοντας, στην παρούσα διπλωματική εργασία αντιμετωπίστηκε το πρόβλημα του διμερούς τηλεχειρισμού σε συστήματα ρομποτικών βραχιόνων τύπου Ηγέτη-Ακόλουθου με άγνωστες και χρονικά μεταβαλλόμενες καθυστερήσεις στο ψηφιακό κανάλι επικοινωνίας και περιορισμούς κατάστασης εξόδου στις αρθρώσεις του Ακόλουθου. Αναπτύχθηκε μια αρχιτεκτονική ελέγχου βασισμένη στον Έλεγχο Προκαθορισμένης Απόδοσης (PPC), που επιτρέπει δυναμική αλλαγή προτεραιοτήτων μεταξύ ακριβούς παρακολούθησης και ασφάλειας, επιλύοντας το ανταγωνιστικό δίλημμα μεταξύ αυτών των δύο στόχων. Η απόδοση του συστήματος αποδείχθηκε συνάρτηση των καθυστερήσεων: όσο μεγαλύτερες οι καθυστερήσεις, τόσο μεγαλύτερα τα εγγυημένα όρια σφάλματος παρακολούθησης, με υποβάθμιση της απόδοσης σε ακραίες περιπτώσεις. Η μαθηματική θεμελίωση του ελέγχου τεκμηριώθηκε πλήρως, και οι εκτενείς προσομοιώσεις σε ρεαλιστικά μοντέλα επιβεβαίωσαν την αποτελεσματική διαχείριση των καθυστερήσεων και την αποφυγή επικίνδυνων ζωνών από τον Ακόλουθο, διατηρώντας την ευστάθεια, την απόδοση και την ασφάλεια του συστήματος, ακόμη και όταν οι καθυστερήσεις υπερέβαιναν τα καθορισμένα όρια.

\bigskip
Παρά τη σημαντική πρόοδο που επιτεύχθηκε, υπάρχουν ακόμα περιθώρια βελτίωσης στο σχεδιασμό ελέγχου για τη διαχείριση πιο πολύπλοκων και δυναμικών περιβαλλόντων. Μελλοντικές μελέτες μπορούν να επεκταθούν από τον χώρο των γωνιών στον κατερσιανό XYZ χώρο, ενώ η απαίτηση για όσο το δυνατόν λιγότερη γνώση αναφορικά με τις διάφοες κλάσεις μη-γραμμικών συστημάτων αποτελεί μείζον θέμα διερεύνησης. Επιπλέον, θα μπορούσαν θα μελετηθούν σενάρια ασυνεχούς και/ή χωρίς γνώση άνω φράγματος χρονικών καθυστερήσεων, περιπτώσεις που μελετήθηκαν στον έλεγχο ευρωστίας αλλά δεν εγγυήθηκε η επιτυχής και μαθηματικά αποδεδειγμένη αντιμετώπισή τους. Τέλος, πέρα από τις προσομοιώσεις με την χρήση λογισμικού MATLAB, θα μπορούσε μελλοντικά να επεκταθεί η παραπάνω μελέτη σε ρεαλιστικούς βραχίονες, εκτελώντας πειράματα με άνθρωπο-χειριστή. 
